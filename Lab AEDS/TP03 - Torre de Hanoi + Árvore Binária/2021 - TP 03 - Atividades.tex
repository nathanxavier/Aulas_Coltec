%----------------------------------------------------------------------------------------------------------------
\documentclass[a4paper]{exam}

\usepackage{fullpage}
\usepackage[brazil]{babel}
\usepackage[utf8]{inputenc}
\usepackage{parskip}
\usepackage{tikz}
\usepackage{amsmath}
\usepackage{amsthm}
\usepackage{amsfonts}
\usepackage{amssymb}
\usepackage{graphicx}
\usepackage{epstopdf}
\usepackage{inputenc}
\usepackage{multicol}
\usepackage{multirow}
\usepackage{array}
\usepackage{subcaption}
\usepackage{float}
\usepackage{enumitem}
\usepackage{geometry}
\usepackage{tabularx}
\usepackage{enumitem}
\usepackage{hyperref}
\usepackage{listings}
\usepackage{xcolor}


%----------------------------------------------------------------------------------------------------------------
% COMANDOS
%----------------------------------------------------------------------------------------------------------------
\geometry{a4paper, margin=2cm}
\renewcommand{\thesubsubsection}{\thesubsection \alph{subsubsection}}

\renewcommand*{\thefootnote}{\fnsymbol{footnote}}
\renewcommand{\theenumi}{\Alph{enumi}}

\pointpoints{ponto}{pontos}
\bonuspointpoints{ponto extra}{pontos extras}
\totalformat{Pergunta \thequestion: \totalpoints pontos}
 
\chqword{Pergunta}
\chpgword{Página}
\chpword{Pontos}
\chbpword{Pontos extra}
\chsword{Pontos Obtidos}
\chtword{Total}

\definecolor{codegreen}{rgb}{0,0.6,0}
\definecolor{codegray}{rgb}{0.5,0.5,0.5}
\definecolor{codepurple}{rgb}{0.58,0,0.82}
\definecolor{backcolour}{rgb}{0.95,0.95,0.92}
\lstdefinestyle{mystyle}{
	backgroundcolor=\color{backcolour},
	commentstyle=\color{codegreen},
	keywordstyle=\color{magenta},
	numberstyle=\tiny\color{codegray},
	stringstyle=\color{codepurple},
	basicstyle=\ttfamily\footnotesize,
	breakatwhitespace=false,
	breaklines=true,
	captionpos=b,
	keepspaces=true,
	numbers=left,
	numbersep=5pt,
	showspaces=false,
	showstringspaces=false,
	showtabs=false, 
	tabsize=2
}
\lstset{style=mystyle}


%----------------------------------------------------------------------------------------------------------------
% TÍTULO
%----------------------------------------------------------------------------------------------------------------
\title{
	\normalfont \normalsize 
	\textsc{Colégio Técnico - UFMG \\ 
		Laboratório de AEDS} \\
	\rule{\linewidth}{0.5pt} \\
	\huge Trabalho Prático 03 \\
	\rule{\linewidth}{2pt} \\
}
\author{\textbf{Aluno:} \hspace{10cm} \textbf{Turma:}}
\date{}

\makeindex

%----------------------------------------------------
% Exercícios
%----------------------------------------------------

\begin{document}
\maketitle

\begin{center}
	\section*{Projeto 01 - Torre de Hanoi}
\end{center}

A Torre de Hanoi é um exercício de lógica muito famoso que envolve o movimento de discos para uma nova localização diferente da inicialmente definida. 

\begin{figure} [H]
	\centering
	\includegraphics[width=.5\textwidth]{Figuras/Hanoi.png}
\end{figure}

Em geral, a solução da Torre de Hanoi está associada à programação recursiva, uma vez que os passos para solução são similares, independente da quantidade de blocos adicionados à torre.

Para a entrega desse projeto, pede-se:

\begin{parts}
	\part Você deve definir cada haste como uma estrutura de pilha diferente.
	\part Segundo a lógica da Torre de Hanoi, um bloco de maior numeração nunca pode estar sobre outro bloco de menor numeração.
	\part É importante sempre mostrar a pilha após cada movimento, para se ter uma noção de como está a dinâmica das torres.
	
	\textbf{Exemplo:}
	$$ \begin{matrix}
	A: & [3 & 2 & 1] \\
	B: & [ &  & ] \\
	C: & [ &  & ]
	\end{matrix} ~~~\rightarrow~~~
	\begin{matrix}
	A: & [3 & 2 & ] \\
	B: & [1 &  & ] \\
	C: & [ &  & ]
	\end{matrix} ~~~\rightarrow~~~
	\begin{matrix}
	A: & [3 &  & ] \\
	B: & [1 &  & ] \\
	C: & [2 &  & ]
	\end{matrix} ~~~\rightarrow~~~
	\begin{matrix}
	A: & [3 &  & ] \\
	B: & [ &  & ] \\
	C: & [2 & 1 & ]
	\end{matrix} $$
	
	\part Você pode optar por duas formas diferentes de apresentação do projeto:
	\begin{enumerate}
		\item A solução para a torre é demonstrada automaticamente, ou;
		\item O usuário pode jogar livremente o jogo. A dinâmica e jogabilidade deve ser definida por você, sendo explicada dentro da programação.
	\end{enumerate}
\end{parts}

\newpage
\begin{center}
	\section*{Projeto 02 - Árvore Binária}
\end{center}

Neste projeto, vamos implementar uma Árvore Binária que será criada de forma aleatória. Para isso, vamos utilizar, novamente, a geração de números aleatórios inteiros entre 0 e 20. Para a criação da árvore, siga os seguintes passos:

\begin{enumerate}
	\item Gere um número inteiro aleatório (\href{https://www.cplusplus.com/reference/cstdlib/rand/}{rand()}).
	\item Desenvolva a árvore alocando o número à esquerda ou à direita, baseado em uma comparação simples de maior ou menor. Caso o número já esteja presente na árvore, ignore e faça o sorteio de um novo número.
	\item Desenvolva esse processo um número limitado de vezes (100 ou 1000 vezes), não sendo necessário alocar todos os número na árvore.
	\item Faça a demonstração da árvore no formato matricial (Matriz de Adjacência). A matriz é criada considerando:
	\begin{itemize}
		\item Cada linha está associada ao número de origem, o nó raiz.
		\item A coluna, por sua vez, está associada às folhas ou aos nós filhos.
		\item Desta forma, cada célula $(x,y)$ da matriz indica a conexão com $x$ sendo a raiz e $y$ a folha.
		\item Exemplo:
		\begin{figure} [H]
			\centering
			\includegraphics[width=.5\textwidth]{Figuras/Tree.png}
		\end{figure}
	
		\begin{table}[H]
			\centering
			\begin{tabular}{cccccccccccccccc}
				& 1 & 2 & 3 & 4 & 5 & 6 & 7 & 8 & 9 & 10 & 11 & 12 & 13 & 14 & 15 \\ \cline{2-16} 
				\multicolumn{1}{c|}{1} & \multicolumn{1}{c|}{} & \multicolumn{1}{c|}{} & \multicolumn{1}{c|}{} & \multicolumn{1}{c|}{} & \multicolumn{1}{c|}{} & \multicolumn{1}{c|}{} & \multicolumn{1}{c|}{} & \multicolumn{1}{c|}{} & \multicolumn{1}{c|}{} & \multicolumn{1}{c|}{} & \multicolumn{1}{c|}{} & \multicolumn{1}{c|}{} & \multicolumn{1}{c|}{} & \multicolumn{1}{c|}{} & \multicolumn{1}{c|}{} \\ \cline{2-16} 
				\multicolumn{1}{c|}{2} & \multicolumn{1}{c|}{} & \multicolumn{1}{c|}{} & \multicolumn{1}{c|}{} & \multicolumn{1}{c|}{} & \multicolumn{1}{c|}{} & \multicolumn{1}{c|}{} & \multicolumn{1}{c|}{} & \multicolumn{1}{c|}{} & \multicolumn{1}{c|}{} & \multicolumn{1}{c|}{} & \multicolumn{1}{c|}{} & \multicolumn{1}{c|}{} & \multicolumn{1}{c|}{} & \multicolumn{1}{c|}{} & \multicolumn{1}{c|}{} \\ \cline{2-16} 
				\multicolumn{1}{c|}{3} & \multicolumn{1}{c|}{1} & \multicolumn{1}{c|}{} & \multicolumn{1}{c|}{} & \multicolumn{1}{c|}{} & \multicolumn{1}{c|}{} & \multicolumn{1}{c|}{1} & \multicolumn{1}{c|}{} & \multicolumn{1}{c|}{} & \multicolumn{1}{c|}{} & \multicolumn{1}{c|}{} & \multicolumn{1}{c|}{} & \multicolumn{1}{c|}{} & \multicolumn{1}{c|}{} & \multicolumn{1}{c|}{} & \multicolumn{1}{c|}{} \\ \cline{2-16} 
				\multicolumn{1}{c|}{4} & \multicolumn{1}{c|}{} & \multicolumn{1}{c|}{} & \multicolumn{1}{c|}{} & \multicolumn{1}{c|}{} & \multicolumn{1}{c|}{} & \multicolumn{1}{c|}{} & \multicolumn{1}{c|}{} & \multicolumn{1}{c|}{} & \multicolumn{1}{c|}{} & \multicolumn{1}{c|}{} & \multicolumn{1}{c|}{} & \multicolumn{1}{c|}{} & \multicolumn{1}{c|}{} & \multicolumn{1}{c|}{} & \multicolumn{1}{c|}{} \\ \cline{2-16} 
				\multicolumn{1}{c|}{5} & \multicolumn{1}{c|}{} & \multicolumn{1}{c|}{} & \multicolumn{1}{c|}{} & \multicolumn{1}{c|}{} & \multicolumn{1}{c|}{} & \multicolumn{1}{c|}{} & \multicolumn{1}{c|}{} & \multicolumn{1}{c|}{} & \multicolumn{1}{c|}{} & \multicolumn{1}{c|}{} & \multicolumn{1}{c|}{} & \multicolumn{1}{c|}{} & \multicolumn{1}{c|}{} & \multicolumn{1}{c|}{} & \multicolumn{1}{c|}{} \\ \cline{2-16} 
				\multicolumn{1}{c|}{6} & \multicolumn{1}{c|}{} & \multicolumn{1}{c|}{} & \multicolumn{1}{c|}{} & \multicolumn{1}{c|}{1} & \multicolumn{1}{c|}{} & \multicolumn{1}{c|}{} & \multicolumn{1}{c|}{1} & \multicolumn{1}{c|}{} & \multicolumn{1}{c|}{} & \multicolumn{1}{c|}{} & \multicolumn{1}{c|}{} & \multicolumn{1}{c|}{} & \multicolumn{1}{c|}{} & \multicolumn{1}{c|}{} & \multicolumn{1}{c|}{} \\ \cline{2-16} 
				\multicolumn{1}{c|}{7} & \multicolumn{1}{c|}{} & \multicolumn{1}{c|}{} & \multicolumn{1}{c|}{} & \multicolumn{1}{c|}{} & \multicolumn{1}{c|}{} & \multicolumn{1}{c|}{} & \multicolumn{1}{c|}{} & \multicolumn{1}{c|}{} & \multicolumn{1}{c|}{} & \multicolumn{1}{c|}{} & \multicolumn{1}{c|}{} & \multicolumn{1}{c|}{} & \multicolumn{1}{c|}{} & \multicolumn{1}{c|}{} & \multicolumn{1}{c|}{} \\ \cline{2-16} 
				\multicolumn{1}{c|}{8} & \multicolumn{1}{c|}{} & \multicolumn{1}{c|}{} & \multicolumn{1}{c|}{1} & \multicolumn{1}{c|}{} & \multicolumn{1}{c|}{} & \multicolumn{1}{c|}{} & \multicolumn{1}{c|}{} & \multicolumn{1}{c|}{} & \multicolumn{1}{c|}{} & \multicolumn{1}{c|}{1} & \multicolumn{1}{c|}{} & \multicolumn{1}{c|}{} & \multicolumn{1}{c|}{} & \multicolumn{1}{c|}{} & \multicolumn{1}{c|}{} \\ \cline{2-16} 
				\multicolumn{1}{c|}{9} & \multicolumn{1}{c|}{} & \multicolumn{1}{c|}{} & \multicolumn{1}{c|}{} & \multicolumn{1}{c|}{} & \multicolumn{1}{c|}{} & \multicolumn{1}{c|}{} & \multicolumn{1}{c|}{} & \multicolumn{1}{c|}{} & \multicolumn{1}{c|}{} & \multicolumn{1}{c|}{} & \multicolumn{1}{c|}{} & \multicolumn{1}{c|}{} & \multicolumn{1}{c|}{} & \multicolumn{1}{c|}{} & \multicolumn{1}{c|}{} \\ \cline{2-16} 
				\multicolumn{1}{c|}{10} & \multicolumn{1}{c|}{} & \multicolumn{1}{c|}{} & \multicolumn{1}{c|}{} & \multicolumn{1}{c|}{} & \multicolumn{1}{c|}{} & \multicolumn{1}{c|}{} & \multicolumn{1}{c|}{} & \multicolumn{1}{c|}{} & \multicolumn{1}{c|}{} & \multicolumn{1}{c|}{} & \multicolumn{1}{c|}{} & \multicolumn{1}{c|}{} & \multicolumn{1}{c|}{} & \multicolumn{1}{c|}{1} & \multicolumn{1}{c|}{} \\ \cline{2-16} 
				\multicolumn{1}{c|}{11} & \multicolumn{1}{c|}{} & \multicolumn{1}{c|}{} & \multicolumn{1}{c|}{} & \multicolumn{1}{c|}{} & \multicolumn{1}{c|}{} & \multicolumn{1}{c|}{} & \multicolumn{1}{c|}{} & \multicolumn{1}{c|}{} & \multicolumn{1}{c|}{} & \multicolumn{1}{c|}{} & \multicolumn{1}{c|}{} & \multicolumn{1}{c|}{} & \multicolumn{1}{c|}{} & \multicolumn{1}{c|}{} & \multicolumn{1}{c|}{} \\ \cline{2-16} 
				\multicolumn{1}{c|}{12} & \multicolumn{1}{c|}{} & \multicolumn{1}{c|}{} & \multicolumn{1}{c|}{} & \multicolumn{1}{c|}{} & \multicolumn{1}{c|}{} & \multicolumn{1}{c|}{} & \multicolumn{1}{c|}{} & \multicolumn{1}{c|}{} & \multicolumn{1}{c|}{} & \multicolumn{1}{c|}{} & \multicolumn{1}{c|}{} & \multicolumn{1}{c|}{} & \multicolumn{1}{c|}{} & \multicolumn{1}{c|}{} & \multicolumn{1}{c|}{} \\ \cline{2-16} 
				\multicolumn{1}{c|}{13} & \multicolumn{1}{c|}{} & \multicolumn{1}{c|}{} & \multicolumn{1}{c|}{} & \multicolumn{1}{c|}{} & \multicolumn{1}{c|}{} & \multicolumn{1}{c|}{} & \multicolumn{1}{c|}{} & \multicolumn{1}{c|}{} & \multicolumn{1}{c|}{} & \multicolumn{1}{c|}{} & \multicolumn{1}{c|}{} & \multicolumn{1}{c|}{} & \multicolumn{1}{c|}{} & \multicolumn{1}{c|}{} & \multicolumn{1}{c|}{} \\ \cline{2-16} 
				\multicolumn{1}{c|}{14} & \multicolumn{1}{c|}{} & \multicolumn{1}{c|}{} & \multicolumn{1}{c|}{} & \multicolumn{1}{c|}{} & \multicolumn{1}{c|}{} & \multicolumn{1}{c|}{} & \multicolumn{1}{c|}{} & \multicolumn{1}{c|}{} & \multicolumn{1}{c|}{} & \multicolumn{1}{c|}{} & \multicolumn{1}{c|}{} & \multicolumn{1}{c|}{} & \multicolumn{1}{c|}{1} & \multicolumn{1}{c|}{} & \multicolumn{1}{c|}{} \\ \cline{2-16} 
				\multicolumn{1}{c|}{15} & \multicolumn{1}{c|}{} & \multicolumn{1}{c|}{} & \multicolumn{1}{c|}{} & \multicolumn{1}{c|}{} & \multicolumn{1}{c|}{} & \multicolumn{1}{c|}{} & \multicolumn{1}{c|}{} & \multicolumn{1}{c|}{} & \multicolumn{1}{c|}{} & \multicolumn{1}{c|}{} & \multicolumn{1}{c|}{} & \multicolumn{1}{c|}{} & \multicolumn{1}{c|}{} & \multicolumn{1}{c|}{} & \multicolumn{1}{c|}{} \\ \cline{2-16} 
			\end{tabular}
		\end{table}
	\end{itemize}
\end{enumerate}




%-------------------------------------------------
\end{document}