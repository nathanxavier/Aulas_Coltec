%----------------------------------------------------------------------------------------------------------------
\documentclass[a4paper]{exam}

\usepackage{fullpage}
\usepackage[brazil]{babel}
\usepackage[utf8]{inputenc}
\usepackage{parskip}
\usepackage{tikz}
\usepackage{amsmath}
\usepackage{amsthm}
\usepackage{amsfonts}
\usepackage{amssymb}
\usepackage{graphicx}
\usepackage{epstopdf}
\usepackage{inputenc}
\usepackage{multicol}
\usepackage{multirow}
\usepackage{array}
\usepackage{subcaption}
\usepackage{float}
\usepackage{enumitem}
\usepackage{geometry}
\usepackage{tabularx}
\usepackage{enumitem}
\usepackage{hyperref}
\usepackage{listings}
\usepackage{xcolor}


%----------------------------------------------------------------------------------------------------------------
% COMANDOS
%----------------------------------------------------------------------------------------------------------------
\geometry{a4paper, margin=2cm}
\renewcommand{\thesubsubsection}{\thesubsection \alph{subsubsection}}

\renewcommand*{\thefootnote}{\fnsymbol{footnote}}
\renewcommand{\theenumi}{\Alph{enumi}}

\pointpoints{ponto}{pontos}
\bonuspointpoints{ponto extra}{pontos extras}
\totalformat{Pergunta \thequestion: \totalpoints pontos}
 
\chqword{Pergunta}
\chpgword{Página}
\chpword{Pontos}
\chbpword{Pontos extra}
\chsword{Pontos Obtidos}
\chtword{Total}

\definecolor{codegreen}{rgb}{0,0.6,0}
\definecolor{codegray}{rgb}{0.5,0.5,0.5}
\definecolor{codepurple}{rgb}{0.58,0,0.82}
\definecolor{backcolour}{rgb}{0.95,0.95,0.92}
\lstdefinestyle{mystyle}{
	backgroundcolor=\color{backcolour},
	commentstyle=\color{codegreen},
	keywordstyle=\color{magenta},
	numberstyle=\tiny\color{codegray},
	stringstyle=\color{codepurple},
	basicstyle=\ttfamily\footnotesize,
	breakatwhitespace=false,
	breaklines=true,
	captionpos=b,
	keepspaces=true,
	numbers=left,
	numbersep=5pt,
	showspaces=false,
	showstringspaces=false,
	showtabs=false, 
	tabsize=2
}
\lstset{style=mystyle}


%----------------------------------------------------------------------------------------------------------------
% TÍTULO
%----------------------------------------------------------------------------------------------------------------
\title{
	\normalfont \normalsize 
	\textsc{Colégio Técnico - UFMG \\ 
		Laboratório de AEDS} \\
	\rule{\linewidth}{0.5pt} \\
	\huge Trabalho Prático 02 \\
	\rule{\linewidth}{2pt} \\
}
\author{\textbf{Aluno:} \hspace{10cm} \textbf{Turma:}}
\date{}

\makeindex

%----------------------------------------------------
% Exercícios
%----------------------------------------------------

\begin{document}
\maketitle

\begin{center}
	\section*{Projeto 01 - Campo Minado}
\end{center}

A partir do programa criado para fazer um Campo Minado, desenvolva um programa, usando Lista Encadeada, que seja capaz de indicar caminhos que ligam a primeira posição da matriz (0,0), até a última posição (n,n) de uma matriz quadrada. Siga as seguintes instruções para descrição desse caminho:

\begin{parts}
	\part Em um primeiro programa, utilize a Diagonal Principal da matriz para descrever a Lista Encadeada. Ao final do programa, indique quantas bombas foram atravessadas neste percurso.
	\part Em um segundo programa, o caminho deve deve ser tracejado de modo ortogonal, podendo somente andar para os lados ou para cima e para baixo. Defina uma trajetória que ande nos cantos toda o caminho para o lado e, depois, na vertical, até chegar ao final da matriz. Indique o número de bombas que foram atravessadas.
	\part Com base no mesmo problema anterior, tente implementar o caminho de forma dinâmica, de modo que, se uma célula de bomba for encontrada, a célula da vertical seja colocada no lugar, tentando chegar até o final da matriz. Se caso a célula na vertical também for uma bomba, finalize o programa indicando que foi incapaz de alcançar o fim da matriz.
	\part (Extra) Implemente o código anterior de modo que, caso o caminho esteja bloqueado, você troque o caminho percorrido da última célula que apresentou um percurso válido.
	
	\begin{table}[H]
		\centering
		\begin{tabular}{|c|c|c|c|}
			\hline
			0,0 & 0,1 & 0,2 & * \\ \hline
			1,0 & 1,1 & * & \\ \hline
			2,0 & * &  & \\ \hline
			3,0 & 3,1 &  & \\ \hline
		\end{tabular}
	\end{table}
	Para exemplificar, o programa, no campo acima, deveria perceguir, em sequência, as células:
	\begin{itemize}
		\item $0,0 \rightarrow 0,1 \rightarrow 0,2$.
		
		Verifica a impossibilidade da última célula e define o caminho:
		\item $0,0 \rightarrow 0,1 \rightarrow 1,1$
		
		Novamente, preso, cria:
		\item $0,0 \rightarrow 1,0 \rightarrow 1,1$;
		
		Percebe que todos os caminhos levam a Roma e troca de novo para:
		\item $0,0 \rightarrow 1,0 \rightarrow 2,0 \rightarrow 3,0 \rightarrow 3,1$, etc.
	\end{itemize}
\end{parts}

\newpage
\begin{center}
	\section*{Projeto 02 - Gestão de caixa com atendimento Normal e Prioritária}
\end{center}

Considere desenvolver um sistema que define qual será a ordem de atendimento em um único caixa. Para isso, existem 2 listas distintas com as informações de atendimento, uma Normal e outra Prioritária. Interno na lista, cada pessoa recebe também um número, referente ao tempo médio que o atendimento vai durar, variando entre 0 a 5.

Em uma única Lista Encadeada, adicione o atendimento dessas pessoas de modo que as pessoas prioritárias sejam sempre colocadas primeiro, mas, sempre que a soma do número de pessoas ou o tempo for maior que 5, é necessário que haja um atendimento normal.
%-------------------------------------------------
\end{document}