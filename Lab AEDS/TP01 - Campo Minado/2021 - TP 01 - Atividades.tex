%----------------------------------------------------------------------------------------------------------------
\documentclass[a4paper]{exam}

\usepackage{fullpage}
\usepackage[brazil]{babel}
\usepackage[utf8]{inputenc}
\usepackage{parskip}
\usepackage{tikz}
\usepackage{amsmath}
\usepackage{amsthm}
\usepackage{amsfonts}
\usepackage{amssymb}
\usepackage{graphicx}
\usepackage{epstopdf}
\usepackage{inputenc}
\usepackage{multicol}
\usepackage{multirow}
\usepackage{array}
\usepackage{subcaption}
\usepackage{float}
\usepackage{enumitem}
\usepackage{geometry}
\usepackage{tabularx}
\usepackage{enumitem}
\usepackage{hyperref}
\usepackage{listings}
\usepackage{xcolor}


%----------------------------------------------------------------------------------------------------------------
% COMANDOS
%----------------------------------------------------------------------------------------------------------------
\geometry{a4paper, margin=2cm}
\renewcommand{\thesubsubsection}{\thesubsection \alph{subsubsection}}

\renewcommand*{\thefootnote}{\fnsymbol{footnote}}
\renewcommand{\theenumi}{\Alph{enumi}}

\pointpoints{ponto}{pontos}
\bonuspointpoints{ponto extra}{pontos extras}
\totalformat{Pergunta \thequestion: \totalpoints pontos}
 
\chqword{Pergunta}
\chpgword{Página}
\chpword{Pontos}
\chbpword{Pontos extra}
\chsword{Pontos Obtidos}
\chtword{Total}

\definecolor{codegreen}{rgb}{0,0.6,0}
\definecolor{codegray}{rgb}{0.5,0.5,0.5}
\definecolor{codepurple}{rgb}{0.58,0,0.82}
\definecolor{backcolour}{rgb}{0.95,0.95,0.92}
\lstdefinestyle{mystyle}{
	backgroundcolor=\color{backcolour},
	commentstyle=\color{codegreen},
	keywordstyle=\color{magenta},
	numberstyle=\tiny\color{codegray},
	stringstyle=\color{codepurple},
	basicstyle=\ttfamily\footnotesize,
	breakatwhitespace=false,
	breaklines=true,
	captionpos=b,
	keepspaces=true,
	numbers=left,
	numbersep=5pt,
	showspaces=false,
	showstringspaces=false,
	showtabs=false, 
	tabsize=2
}
\lstset{style=mystyle}


%----------------------------------------------------------------------------------------------------------------
% TÍTULO
%----------------------------------------------------------------------------------------------------------------
\title{
	\normalfont \normalsize 
	\textsc{Colégio Técnico - UFMG \\ 
		Laboratório de AEDS} \\
	\rule{\linewidth}{0.5pt} \\
	\huge Trabalho Prático 01 \\
	\rule{\linewidth}{2pt} \\
}
\author{\textbf{Aluno:} \hspace{10cm} \textbf{Turma:}}
\date{}

\makeindex

%----------------------------------------------------
% Exercícios
%----------------------------------------------------

\begin{document}
\maketitle

\begin{center}
	\section*{Projeto 01 (Revisão) - Campo Minado}
\end{center}

A proposta será criar a base para um campo minado, de modo que, pela programação, seja desenvolvida a matriz com as bombas e o números. Não é necessário implementar a jogabilidade.

O desenvolvimento pode ser feito em etapas, de modo a facilitar a criação do jogo. O objetivo deste projeto é:

\begin{itemize}
	\item Apresentar uma matriz de tamanho (\textbf{x, y}) variados que possua um número (\textbf{n}) de bombas também variável. Tanto o tamanho quanto o número de bombas deve ser informado pelo usuário.
	\item Dentro desta matriz, você deve alocar as (\textbf{n}) bombas em posições aleatórias da matriz. Para isso, utilize a função (\href{https://www.cplusplus.com/reference/cstdlib/rand/}{rand()}).
	\item Após a alocação das bombas, a matriz deve ser preenchida com os números, de modo que, cada célula, indica a quantidade de bombas que existe ao redor. Vide exemplos:
\end{itemize}

\centering
\textbf{Exemplo 1: Tamanho= 3x3, Bomba=1}
\begin{table}[H]
	\centering
	\begin{tabular}{|c|c|c|}
		\hline
		1 & 1 & 1 \\ \hline
		1 & * & 1 \\ \hline
		1 & 1 & 1 \\ \hline
	\end{tabular}
\end{table}

\textbf{Exemplo 2: Tamanho= 3x3, Bomba=2}
\begin{table}[H]
	\centering
	\begin{tabular}{|c|c|c|}
		\hline
		* & 2 & 1 \\ \hline
		1 & 2 & * \\ \hline
		& 1 & 1 \\ \hline
	\end{tabular}
\end{table}

\textbf{Exemplo 3: Tamanho= 4x5, Bomba=3}
\begin{table}[H]
	\centering
	\begin{tabular}{|c|c|c|c|c|}
		\hline
		1 & * & 2 & 1 &  \\ \hline
		1 & 2 & * & 1 &  \\ \hline
		& 2 & 2 & 2 &  \\ \hline
		& 1 & * & 1 &  \\ \hline
	\end{tabular}
\end{table}

\textbf{Exemplo 4: Tamanho= 5x5, Bomba=7}
\begin{table}[H]
	\centering
	\begin{tabular}{|c|c|c|c|c|}
		\hline
		* & 1 &   & 1 & * \\ \hline
		2 & 3 & 1 & 3 & 2 \\ \hline
		* & 2 & * & 2 & * \\ \hline
		2 & 3 & 3 & 3 & 1 \\ \hline
		1 & * & 2 & * & 1 \\ \hline
	\end{tabular}
\end{table}

%-------------------------------------------------
\end{document}