%----------------------------------------------------------------------------------------------------------------
\documentclass[a4paper]{exam}

\usepackage{fullpage}
\usepackage[brazil]{babel}
\usepackage[utf8]{inputenc}
\usepackage{parskip}
\usepackage{tikz}
\usepackage{amsmath}
\usepackage{amsthm}
\usepackage{amsfonts}
\usepackage{amssymb}
\usepackage{graphicx}
\usepackage{epstopdf}
\usepackage{inputenc}
\usepackage{multicol}
\usepackage{multirow}
\usepackage{array}
\usepackage{subcaption}
\usepackage{float}
\usepackage{enumitem}
\usepackage{geometry}
\usepackage{tabularx}
\usepackage{enumitem}
\usepackage{hyperref}
\usepackage{listings}
\usepackage{xcolor}


%----------------------------------------------------------------------------------------------------------------
% COMANDOS
%----------------------------------------------------------------------------------------------------------------
\geometry{a4paper, margin=2cm}
\renewcommand{\thesubsubsection}{\thesubsection \alph{subsubsection}}

\renewcommand*{\thefootnote}{\fnsymbol{footnote}}
\renewcommand{\theenumi}{\Alph{enumi}}

\pointpoints{ponto}{pontos}
\bonuspointpoints{ponto extra}{pontos extras}
\totalformat{Pergunta \thequestion: \totalpoints pontos}
 
\chqword{Pergunta}
\chpgword{Página}
\chpword{Pontos}
\chbpword{Pontos extra}
\chsword{Pontos Obtidos}
\chtword{Total}

\definecolor{codegreen}{rgb}{0,0.6,0}
\definecolor{codegray}{rgb}{0.5,0.5,0.5}
\definecolor{codepurple}{rgb}{0.58,0,0.82}
\definecolor{backcolour}{rgb}{0.95,0.95,0.92}
\lstdefinestyle{mystyle}{
	backgroundcolor=\color{backcolour},
	commentstyle=\color{codegreen},
	keywordstyle=\color{magenta},
	numberstyle=\tiny\color{codegray},
	stringstyle=\color{codepurple},
	basicstyle=\ttfamily\footnotesize,
	breakatwhitespace=false,
	breaklines=true,
	captionpos=b,
	keepspaces=true,
	numbers=left,
	numbersep=5pt,
	showspaces=false,
	showstringspaces=false,
	showtabs=false, 
	tabsize=2
}
\lstset{style=mystyle}


%----------------------------------------------------------------------------------------------------------------
% TÍTULO
%----------------------------------------------------------------------------------------------------------------
\title{
	\normalfont \normalsize 
	\textsc{Colégio Técnico - UFMG \\ 
		Laboratório de AEDS} \\
	\rule{\linewidth}{0.5pt} \\
	\huge Trabalho Prático 04 \\
	\rule{\linewidth}{2pt} \\
}
\author{\textbf{Aluno:} \hspace{10cm} \textbf{Turma:}}
\date{}

\makeindex

%----------------------------------------------------
% Exercícios
%----------------------------------------------------

\begin{document}
\maketitle

\begin{center}
	\section*{Criação de um Labirinto}
\end{center}

Como visto na parte teórica, os Grafos são similares às listas com as seguintes informações implementadas:
\begin{itemize}
	\item Origem;
	\item Destino, e;
	\item Peso.
\end{itemize}

O conceito de Grafos pode, facilmente, ser observado em situações como o labirinto apresentado abaixo, onde, existem vértices interligados.
\begin{figure}[H]
	\centering
	\includegraphics[width=.3\textwidth]{Figuras/Maze.png}
\end{figure}

Neste projeto, vamos desenvolver a técnica chamada Rapidly-Exploring Random Trees (RRT). Maiores informações podem ser vistas nos seguintes links:
\begin{itemize}
	\item \href{https://www.youtube.com/watch?v=Ob3BIJkQJEw}{\textbf{RRT, RRT* \& Random Trees}}
	\item \href{https://www.wolframcloud.com/objects/demonstrations/RapidlyExploringRandomTreeRRTAndRRT-source.nb}{\textbf{Implementação Wolfram Notebook}}
\end{itemize}

Para a implementação da técnica, siga as seguintes etapas:

\begin{enumerate}
	\item Construa \textbf{Vértices} que possuam como informações de Posição (X,Y).
	
	\item Similar a um Grafo, cada vértice também deve apontar para outro.
	
	\item No programa principal, crie vértices em posições aleatórios.
	
	\item Faça uma busca por todos os nós do mapa para encontrar qual a menor distância entre o vértice novo e outro já dentro do mapa.
	
	Para isso, você pode aplicar o seguinte cálculo entre as posições dos vértices:
	
	$$ \text{dist}= \sqrt{(X_i -X_\text{Novo})^2 + (Y_i -Y_\text{Novo})^2} $$
	
	\item Encontrada a menor distância, faça um link entre o vértice novo e o vértice mais próximo.
\end{enumerate}

%-------------------------------------------------
\end{document}